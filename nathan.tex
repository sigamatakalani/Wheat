The purpose of the SRS is to assist students, lecturers, or anyone who may need help manoeuvring about campus towards a certain goal or help them better know the aspects of campus. This involves giving them navigation points towards places of interest, this can be classes, events that take place on campus, and transportation routes such as Tuks bus drop off and pick off locations, restaurants, and other things that requirements.

The product name of this project is NavUP aka Navigation UP.
The product will be able to work in sync with foreign systems, that include a large variety of device platforms which will intend to use it which include IOS, Android as the app will be predominately mobile device focused.
The product will be able to give you a detailed description of where your current location is using three different methods as follows:
*Wi-Fi by using different Wi-Fi connection points one would be able to use the strengths of them to give a approximation of where you are in doors.
*Cell phone towers will be used in substitute in the event that Wi-Fi is not available or if the person wants to use the site but is close proximity of the varsity but not on it.
*GPS will be used in the event that a student is on a bus towards another campus, in order to triangulate their position of the location as they are on route towards the campus.

Navigation through the buildings will be done also influenced by interpretation of building schematics,
The app will be able to provide people with the ability to see time of arrival at their goal, calculate routes to use based on user preference in categories of fastest routes, safest routes, routes with the least traffic and also inform users of unpredictable changes on campus such as building collapses by notifying users to not go that route. Users will also be able to set reminders on their devices reminding them to go to a particular in a time or even alter a route by appending ‘pit stops’ which will be errands that will be done by the clients and the routes will be recalculated and displayed incorporating showing these things.
Clients will be able to login using multiple devices after they had already been registered in which third party services will be able to promote events, items and other things they wish by logging on and pushing them onto the app’s calendar.

Benefits of this include greater productivity as a person’s time table can also be incorporated to generate a weekly schedule for students who have classes in specific locations and how to navigate them.
The app will also cater for those of physical impairments such as heavily using text for those who have hearing issues, for those with eye disabilities the device will have different forms of design schematics to choose from, as well  


