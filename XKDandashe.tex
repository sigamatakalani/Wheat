\textbf{User Interfaces}
The user interfaces for the software shall be compatible with devices that are able to connect to the University of Pretoria's Wifi hotspots. Therefore devices like mobile phones and tablets that use. The user will be able to set their starting position whether they set it themselves or use the wifi connection to determine their current position. As one can see in fig.1, we see that the user can use this interface to set a destination and also use previous destinations in case they travelling to the same place. 

Once the user is ready to search they are transferred to fig. 2 where they can choose their desired route which they feel most comfortable with. Options are given to the user to select which route they prefer, also giving an optimal route to get the user to their destination the quickest and shortest way possible with the least traffic. Once the route is selected the user can choose to begin the navigation.

A visual representation,fig. 3, of the location of the user is given while they are enroute to their destination. Push notifications are sent to the user to let them know which direction should take. Total steps, distance and calories are taken to determine how long the user took to reach their destinations, fig 4, and these statistics are used to give the user goals to achieve. A timetable integration,fig. 5, will be included which will give the user a "Take me to class" option that will keep track of the class venues and times.


\textbf{Software Interfaces}
The NavUP application communicates with the connected wifi router in order to get geographical information about where the user is located and then visually represents it on the user's device. Once the current location is found, the application waits for the user to input a target destination then starts to calculate routes. 

During calculation all possible routes are determined by discerning pathways, checking most travelled route and checking pedestrian traffic. The user is then given the option to select the route they want to take, once the route is selected then navigation begins. Directions are given to the user as they are enroute and their walking statistics are recorded for their profile.


\textbf{Memory}
The NavUP application has some restrictions about the resource allocation. To avoid problems with overloading the operating system, it would be safe to assume that the application is only allowed to use 25 megabytes while running and the maximum amount of hard drive space would be 30 megabytes.


\textbf{Site Adaptation Requirements}
